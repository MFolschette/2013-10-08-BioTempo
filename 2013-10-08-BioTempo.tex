\documentclass[fleqn,8pt,t]{beamer}

\usepackage[english]{babel}
\usepackage[utf8]{inputenc}
\usepackage[T1]{fontenc}
%\usepackage{french} % Sommaire en début de document
%\usepackage[top=2cm, bottom=2cm, left=2cm, right=2cm]{geometry} % Marges

\usepackage{amsmath} % Maths
\usepackage{amsfonts} % Maths
\usepackage{amssymb} % Maths
\usepackage{stmaryrd} % Maths (crochets doubles)

%\usepackage{listings} % Mise en forme du code (pour Hoare) ## À REVOIR ###
%\usepackage{ifthen} % Structures If Then Else
\usepackage{theorem} % Styles supplémentaires pour théorèmes
\usepackage{url}
\usepackage{array}  % Tableaux évolués
\usepackage{multirow}  % Pour des colonnes sur plusieurs lignes

%\usepackage{enumerate} % Changer les puces des listes d'énumération
%\usepackage{setspace} % Changer les interlignes

%\usepackage{subfig} % Créer des sous-figures
%\usepackage{graphicx} % Importer des images

\usepackage{ulem}  % Pour l'attribut barré

\usepackage{comment}

% Police
\usepackage{lmodern}
%\usepackage{libertine}


%%%%%%%%%%%%%%%%%%%%%%%%%%%%%%%%%%%%%%
\usepackage{tikz}
\newdimen\pgfex
\newdimen\pgfem
\usetikzlibrary{arrows,shapes,shadows,scopes}
\usetikzlibrary{positioning}
\usetikzlibrary{matrix}
\usetikzlibrary{decorations.text}
\usetikzlibrary{decorations.pathmorphing}

\input{../macros}
\input{../macros-ph}
\input{../macros-abstr}

\def\Pint{\textsc{PINT}}
\def\PH{\mathcal{PH}}

\tikzstyle{sort}=[fill=lightgray, rounded corners, draw=black]
\tikzstyle{process}=[circle,draw,minimum size=15pt,font=\footnotesize,inner sep=1pt]
\tikzstyle{black process}=[process, draw=blue, fill=red,text=black,font=\bfseries]
\tikzstyle{highlighted process}=[current process, fill=gray]
\tikzstyle{process box}=[fill=none,draw=black,rounded corners]
\tikzstyle{current process}=[process,fill=blue]
\tikzstyle{tick label}=[font=\footnotesize]
\tikzstyle{tick}=[densely dotted]
\tikzstyle{hit}=[->,>=angle 45]
\tikzstyle{selfhit}=[min distance=30pt,curve to]
\tikzstyle{bounce}=[densely dotted,>=stealth',->]
\tikzstyle{hlhit}=[very thick]
\tikzstyle{ulhit}=[draw=lightgray,fill=lightgray]
\tikzstyle{pulhit}=[fill=lightgray]
\tikzstyle{bulhit}=[draw=lightgray]

\tikzstyle{hitless graph}=[every edge/.style={draw=red,-}]

\tikzstyle{aS}=[every edge/.style={draw,->}]
\tikzstyle{Asol}=[draw,circle,minimum size=5pt,inner sep=0]
\tikzstyle{Aproc}=[draw]
\tikzstyle{Aobj}=[]

\renewcommand{\TState}[2]{
  \foreach \proc in {#2} {
        \only<#1>{ \node[current process] (\proc) at (\proc.center) {}; }
  };
}

%\definecolor{darkred}{rgb}{0.5,0,0}
\definecolor{lightred}{rgb}{1,0.8,0.8}
\definecolor{lightgreen}{rgb}{0.7,1,0.7}
\definecolor{darkgreen}{rgb}{0,0.5,0}
\definecolor{darkblue}{rgb}{0,0,0.5}
\definecolor{darkyellow}{rgb}{0.5,0.5,0}
\definecolor{lightyellow}{rgb}{1,1,0.6}
\definecolor{darkcyan}{rgb}{0,0.6,0.6}
\definecolor{darkorange}{rgb}{0.8,0.2,0}

\definecolor{notsodarkgreen}{rgb}{0,0.7,0}

%\definecolor{coloract}{rgb}{0,1,0}
%\definecolor{colorinh}{rgb}{1,0,0}
\colorlet{coloract}{darkgreen}
\colorlet{colorinh}{red}
\colorlet{coloractgray}{lightgreen}
\colorlet{colorinhgray}{lightred}
\colorlet{colorinf}{darkgray}
\colorlet{coloractgray}{lightgreen}
\colorlet{colorinhgray}{lightred}

\colorlet{colorgray}{lightgray}


\tikzstyle{grn}=[every node/.style={circle,draw=black,outer sep=2pt,minimum
                size=15pt,text=black}, node distance=1.5cm]
\tikzstyle{inh}=[>=|,-|,draw=colorinh,thick, text=black,label]
\tikzstyle{act}=[->,>=triangle 60,draw=coloract,thick,color=coloract]
\tikzstyle{inhgray}=[>=|,-|,draw=colorinhgray,thick, text=black,label]
\tikzstyle{actgray}=[->,>=triangle 60,draw=coloractgray,thick,color=coloractgray]
\tikzstyle{inf}=[->,draw=colorinf,thick,color=colorinf]
%\tikzstyle{elabel}=[fill=none, above=-1pt, sloped,text=black, minimum size=10pt, outer sep=0, font=\scriptsize,draw=none]
\tikzstyle{elabel}=[fill=none,text=black, above=-2pt,%sloped,
minimum size=10pt, outer sep=0, font=\scriptsize, draw=none]
%\tikzstyle{elabel}=[]


\tikzstyle{plot}=[every path/.style={-}]
\tikzstyle{axe}=[gray,->,>=stealth']
\tikzstyle{ticks}=[font=\scriptsize,every node/.style={gray}]
\tikzstyle{mean}=[thick]
\tikzstyle{interval}=[line width=5pt,red,draw opacity=0.7]
\definecolor{lightred}{rgb}{1,0.3,0.3}

\tikzstyle{hl}=[yellow]
\tikzstyle{hl2}=[orange]

\tikzstyle{every matrix}=[ampersand replacement=\&]
\tikzstyle{shorthandoff}=[]
\tikzstyle{shorthandon}=[]
%%%%%%%%%%%%%%%%%%%%%%%%%%%%%%%%%%%%%%%%



% Commande À FAIRE
\usepackage{color} % Couleurs du texte
%\newcommand{\afaire}[1]{\textcolor{red}{[À FAIRE : #1]}}
\newcommand{\todo}[1]{\textcolor{red}{<[[#1]]>}}



\colorlet{couleurtheme}{gray}  % Couleur principale du thème
\colorlet{couleurcit}{gray}  % Couleur des citations
\colorlet{couleurex}{blue}  % Couleur des citations
\colorlet{couleurliens}{darkblue}  % Couleur des citations

\usetheme{Pittsburgh}   % Thème général
\usefonttheme{default}  % Thème de polices
\setbeamertemplate{navigation symbols}{}  % Pas de menu de navigation
%\setbeamertemplate{itemize item}[x]   % Puces des listes

\usecolortheme[named=couleurtheme]{structure}    % Couleur de la structure : titres et puces
%\setbeamercolor{normal text}{bg=black,fg=white}  % Couleur du texte
\setbeamercolor{item}{fg=couleurtheme}           % Couleur des puces
%\setbeamercolor{item projected}{fg=black}        % Couleur des recouvrements
%\setbeamercolor{alerted text}{fg=yellow}         % ?

\setbeamerfont{frametitle}{size=\Large}  % Police des titres


% Flèche grise
\newcommand{\f}{\textcolor{couleurtheme}{\textbf{$\rightarrow$\ }}}
\newcommand{\cth}[1]{\textcolor{couleurtheme}{#1}}

% Environnement liste avec flèches
\newenvironment{fleches}{%
\begin{list}{}{%
\setlength{\labelwidth}{1em}% largeur de la boîte englobant le label
\setlength{\labelsep}{0pt}% espace entre paragraphe et l’étiquette
%\setlength{\itemsep}{1pt}
%\setlength{\leftmargin}{\labelwidth+\labelsep}% marge de gauche
\renewcommand{\makelabel}{\f}%
}}{\end{list}}

% Liste sans puce
\newenvironment{liste}{%
\begin{list}{}{%
\setlength{\labelwidth}{0em}% largeur de la boîte englobant le label
\setlength{\labelsep}{0pt}% espace entre paragraphe et l’étiquette
\setlength{\leftmargin}{0em}% marge de gauche
%\renewcommand{\makelabel}{\f}%
}}{\end{list}}

% Style des exemples
\newcommand{\ex}[1]{\textcolor{couleurex}{#1}}
\newcommand{\qex}[1]{\quad \ex{#1}}
\newcommand{\rex}[1]{\hfill \ex{#1}}
\newcommand{\redex}[1]{\textcolor{red}{#1}}

\newcommand{\lien}[1]{\textcolor{couleurliens}{\underline{\url{#1}}}}

\newcommand{\console}[1]{\textcolor{darkgray}{#1}}

% Style des citations
\newcommand{\tscite}[1]{\textcolor{couleurcit}{#1}}
\newcommand{\tcite}[1]{\textcolor{couleurcit}{[#1]}}

\newcommand{\cpmrtcsb}{Paulevé, Magnin, Roux in Transactions on Computational Systems Biology, 2011}
\newcommand{\cpmrmscs}{Paulevé, Magnin, Roux in Mathematical Structures in Computer Science, 2012}
\newcommand{\cfpimrcmsb}{{\scriptsize Folschette, Paulevé, Inoue, Magnin, Roux in Computational Methods in Systems Biology, 2012}}
\newcommand{\crcbmfma}{Richard, Comet, Bernot in Modern Formal Methods and App., 2006}


% Style de texte mis en valeur
\newcommand{\tval}[1]{\textbf{#1}}

% Un vrai symbole pour l'ensemble vide
\renewcommand{\emptyset}{\varnothing}

% Pour définir la conférence et son nom court
\newcommand{\conference}[2]{\def\theconference{#2}
\def\insertshortconference{\ifthenelse{\equal{#1}{-}}{#2}{\ifthenelse{\equal{#1}{}}{#2}{#1}}}}



\newcommand{\thedate}{2013/03/27}
\date{\thedate}
\conference{ASSB'13/Student workshop}{Advances in Systems and Synthetic Biology\\Modelling Complex Biological Systems in the Context of Genomics\\Thematic Research School 2013\\--- Student workshop ---}
\title[Introduction to the PH and inference of its underlying BRN]{Introduction to the Process Hitting and inference of its underlying Biological Regulatory Network}
\author{Maxime FOLSCHETTE}




\setbeamertemplate{footline}{\color{gray}%
\scriptsize
\quad\strut%
\insertauthor%
\hfill%
\insertframenumber/\inserttotalframenumber%
\hfill%
\insertshortconference{} --- \thedate\quad\strut
}


\newcommand{\headersep}{$\circ$} % \bullet \triangleright

\setbeamertemplate{headline}{\color{gray}%
\vskip0.3em%
\quad\strut%
{\scriptsize\color{black}%
% Gris si une section existe
\ifthenelse{\equal{\thesection}{0}}{}{%
\ifthenelse{\equal{\lastsection}{x}}{}{%
\color{gray}%
}}%
\insertshorttitle
\ifthenelse{\equal{\thesection}{0}}{}{%
\ifthenelse{\equal{\lastsection}{x}}{}{%
~\headersep{} %
% Gris si une sous-section existe
\ifthenelse{\equal{\thesubsection}{0}}{\color{black}}{%
\ifthenelse{\equal{\lastsubsection}{x}}{\color{black}}{%
\color{gray}%
}}%
\insertsectionhead%
%
\ifthenelse{\equal{\thesubsection}{0}}{}{%
\ifthenelse{\equal{\lastsubsection}{x}}{}{%
~\headersep{} \color{black}\insertsubsectionhead%
%
}}}}}%
\vskip-5ex%
}



\def \scaleex {0.85}
\def \scaleinf {0.6}

\colorlet{colorb}{blue}
\colorlet{colora1}{yellow}
\colorlet{colora0}{green}
\colorlet{colora1font}{darkyellow}
\colorlet{colora0font}{darkgreen}

\colorlet{exanswer}{blue}
\colorlet{colorgray}{lightgray}

\definecolor{colortitle}{rgb}{0.54,0.8,0.9}


\begin{document}

\begin{frame}[plain,label=title]

% Cadre de titre
\begin{center}
\vspace{1cm}
\setbeamercolor{postit}{fg=black,bg=colortitle}
\begin{beamercolorbox}[sep=0.5em]{postit}
\centering
\Large
\textbf{%
{\normalsize\theconference{}}\\~\\%
\inserttitle
}
\end{beamercolorbox}

% Auteurs et instituts
\par
\medskip
\bigskip
\normalsize
Maxime FOLSCHETTE

\medskip
\footnotesize
MeForBio / IRCCyN / École Centrale de Nantes (Nantes, France)

\texttt{maxime.folschette@irccyn.ec-nantes.fr}

\url{http://www.irccyn.ec-nantes.fr/~folschet/}

\bigskip
Joint work with:
\\
\normalsize
Loïc PAULEVÉ, Katsumi INOUE, Morgan MAGNIN, Olivier ROUX
\end{center}

\end{frame}


\input{parts/ex.tex}
\newcommand{\cmodels}{\bigskip
\quad\tval{\ex{egfr20}}: \tcite{Epidermal Growth Factor Receptor, by Özgür Sahin \textit{et al.}}\\
\quad\tval{\ex{egfr104}}: \tcite{Epidermal Growth Factor Receptor, by Regina Samaga \textit{et al.}}\\
\quad\tval{\ex{tcrsig40}}: \tcite{T-Cell Receptor Signaling, by Steffen Klamt \textit{et al.}}\\
\quad\tval{\ex{tcrsig94}}: \tcite{T-Cell Receptor Signaling, by Julio Saez-Rodriguez \textit{et al.}}\\}

\section{Introduction}
% Diapo d'intro

\begin{frame}[c]
  \frametitle{Context and Aims}

\tval{MeForBio} team: Algebraic modeling to study complex dynamical biological systems

%\bigskip
\begin{center}
  \includegraphics[height=3.5cm]{figs/dnascheme_white.png}
\end{center}

\pause
\begin{enumerate}[1)]
  \item Two main models
  \begin{itemize}
    \item Historical model: \tval{Biological Regulatory Network (René Thomas)}
    \item New developed model: \tval{Process Hitting}
  \end{itemize}

\medskip
  \item Allow efficient translation from Process Hitting to BRN
\end{enumerate}

\end{frame}


\section{Frameworks Definitions}
\subsection{The Process Hitting framework}
%% Définition du Process Hitting + sortes coopératives

\begin{frame}[t]
  \frametitle{The Process Hitting modeling}
  \framesubtitle{\tcite{\cpmrtcsb}}

% 1 : Sortes
\only<1>{
\tikzstyle{process}=[circle,minimum size=15pt,font=\footnotesize,inner sep=1pt]
\tikzstyle{tick label}=[color=white, font=\footnotesize]
\tikzstyle{tick}=[transparent]
\tikzstyle{hit}=[transparent]
\tikzstyle{selfhit}=[transparent, min distance=30pt,curve to]
\tikzstyle{bounce}=[transparent]
\tikzstyle{hlhit}=[transparent]
\begin{center}\scalebox{\scaleex}{
\begin{tikzpicture}
\exphdef
\end{tikzpicture}
}\end{center}
}

% 2 : Processus
\only<2>{
\tikzstyle{process}=[circle,draw,minimum size=15pt,font=\footnotesize,inner sep=1pt]
\tikzstyle{tick label}=[font=\footnotesize]
\tikzstyle{tick}=[densely dotted]
\tikzstyle{hit}=[transparent]
\tikzstyle{selfhit}=[transparent, min distance=30pt,curve to]
\tikzstyle{bounce}=[transparent]
\tikzstyle{hlhit}=[transparent]
\begin{center}\scalebox{\scaleex}{
\begin{tikzpicture}
\exphdef
\end{tikzpicture}
}\end{center}
}

% 3 : États
\only<3>{
\tikzstyle{hit}=[transparent]
\tikzstyle{selfhit}=[transparent, min distance=30pt,curve to]
\tikzstyle{bounce}=[transparent]
\tikzstyle{hlhit}=[transparent]
\begin{center}\scalebox{\scaleex}{
\begin{tikzpicture}
\exphdef

\TState{3}{a_0,b_1,z_0}
\end{tikzpicture}
}\end{center}
}

% 4 : Actions
\only<4->{
\tikzstyle{tick}=[densely dotted]
\tikzstyle{hit}=[->,>=angle 45]
\tikzstyle{selfhit}=[min distance=30pt,curve to]
\tikzstyle{bounce}=[densely dotted,>=stealth',->]
\tikzstyle{hlhit}=[very thick]
\begin{center}\scalebox{\scaleex}{
\begin{tikzpicture}
\exphdef
\TState{4}{a_0,b_1,z_0}
\TState{5}{a_0,b_1,z_1}
\TState{6}{a_1,b_1,z_1}
\TState{7}{a_1,b_1,z_2}
\end{tikzpicture}
}\end{center}
}

\medskip
\begin{liste}
  \item \tval{Sorts}: components \qex{$a$, $b$, $z$}
\pause[2]
  \item \tval{Processes}: local states / levels of expression \qex{$z_0$, $z_1$, $z_2$}
\pause[3]
  \item \tval{States}: sets of active processes%
  \only<3-4>{\qex{$\PHetat{a_0, b_1, z_0}$}}%
  \only<5>{\qex{$\PHetat{a_0, b_1, z_1}$}}%
  \only<6>{\qex{$\PHetat{a_1, b_1, z_1}$}}%
  \only<7>{\qex{$\PHetat{a_1, b_1, z_2}$}}%
\pause[4]
  \item \tval{Actions}: dynamics \qex{\only<4>{\underline}{$\PHfrappe{b_1}{z_0}{z_1}$}, \only<4-5>{\underline}{$\PHfrappe{a_0}{a_0}{a_1}$}, \only<6>{\underline}{$\PHfrappe{a_1}{z_1}{z_2}$}}
\end{liste}
\end{frame}



\begin{frame}
  \frametitle{Adding cooperations}
  \framesubtitle{\tcite{\cpmrtcsb}}

\begin{center}\scalebox{\scaleex}{
\begin{tikzpicture}
\exphcoop
\end{tikzpicture}
}\end{center}

\medskip
\only<-14>{
\begin{liste}
  \item How to introduce some \tval{cooperation} between sorts? \qex{$\PHfrappe{a_1 \wedge b_0}{z_1}{z_2}$}
\pause[4]
  \item Solution: a \tval{cooperative sort} \qex{$ab$} \only<12->{\quad to express \qex{$a_1 \wedge b_0$}}
\pause[8]
  \item Constraint: each configuration is represented by one process \qex{$\PHetat{a_1,b_0} \pause[11]\Rightarrow ab_{10}$}
\pause[14]
  \item Advantage: regular sort; drawbacks: complexity, temporal shift
\end{liste}}
\end{frame}



\begin{frame}[c]
  \frametitle{The Process Hitting modeling}

\begin{itemize}
  \item \tval{Dynamic} modeling with an \tval{atomistic} point of view
  \begin{fleches}
    \item Independent actions
    \item Cooperation modeled with cooperative sorts
  \end{fleches}

  \smallskip
  \item Efficient \tval{static analysis}
  \begin{fleches}
    \item Reachability of a process can be computed in \tval{polynomial time}\\
          \quad in the number of sorts
  \end{fleches}

  \smallskip
  \item Useful for the study of \tval{large biological models}
  \begin{fleches}
    \item Up to hundreds of sorts
  \end{fleches}

  \smallskip
  \item (Future) extensions
  \begin{fleches}
    \item Actions with priorities
    \item Continuous time with clocks?
  \end{fleches}
\end{itemize}

\end{frame}


\subsection{Thomas Modeling}
%% Définition du modèle de Thomas

\colorlet{light}{colorgray}

\begin{frame}
  \frametitle{Biological Regulatory Network (Thomas' modeling)}
  \framesubtitle{\tcite{\crcbmfma}}

\begin{tabular}{cccc}

\begin{tikzpicture}[grn]
\path[use as bounding box] (-0.7,-0.3) rectangle (2.5,2);
% Nœuds noirs
\only<1,3->{
  \node[inner sep=0] (z) at (2,0.75) {z};
  \node[inner sep=0] (a) at (0,1.5) {a};
  \node[inner sep=0] (b) at (0,0) {b};
  \path
    node[elabel, below=-1em of a] {$0..1$}
    node[elabel, below=-1em of b] {$0..1$}
    node[elabel, below=-1em of z] {$0..2$};}
% Nœuds grisés
\only<2>{
  \node[inner sep=0,light] (z) at (2,0.75) {z};
  \node[inner sep=0,light] (a) at (0,1.5) {a};
  \node[inner sep=0,light] (b) at (0,0) {b};
  \path
    node[elabel, below=-1em of a,light] {$0..1$}
    node[elabel, below=-1em of b,light] {$0..1$}
    node[elabel, below=-1em of z,light] {$0..2$};}

% Arcs colorés
\only<1,4->{\path
  (a) edge[inh,loop left=10] node[elabel, left] {$1-$} (a)
  (a) edge[act] node[elabel, above=-2pt] {$1+$} (z)
  (b) edge[inh] node[elabel, below=-2pt] {$1-$} (z);}
% Arcs grisés
\only<2-3>{\path
  (a) edge[inhgray,loop left=10] node[elabel, left,light] {$1-$} (a)
  (a) edge[actgray] node[elabel, above=-2pt,light] {$1+$} (z)
  (b) edge[inhgray] node[elabel, below=-2pt,light] {$1-$} (z);}
\end{tikzpicture}
&%
\only<2-4>{\color{light}}%
\begin{tabular}[b]{c|c|c}
  \multicolumn{2}{c|}{$\omega$} & \multirow{2}{*}{$k_{z, \omega}$} \\
\cline{1-2}
  $a$ & $b$ & \\
\hline
  $-$ & $+$ & $1$ \\
  $-$ & $-$ & $0$ \\
  $+$ & $+$ & $2$ \\
  \only<6>{\color{red}}$+$ & \only<6>{\color{red}}$-$ & \only<6>{\color{red}}$1$
\end{tabular}
%\begin{tabular}[b]{c|c}
%  $\omega$ & $k_{z, \omega}$ \\
%\hline
%  $\emptyset$ & $1$ \\
%  $\{b\}$ & $0$ \\
%  \only<7-8>{\color{red}}$\{a\}$ & \only<7-8>{\color{red}}$2$ \\
%  $\{a;b\}$ & $1$
%\end{tabular}
~~~~&
\only<2-4>{\color{light}}%
\begin{tabular}[b]{c|c}
  $\omega$ & \multirow{2}{*}{$k_{a, \omega}$} \\
\cline{1-1}
  $a$ & \\
\hline
  $+$ & $1$ \\
  $-$ & $0$
\end{tabular}
%\begin{tabular}[b]{c|c}
%  $\omega$ & $k_{a, \omega}$ \\
%\hline
%  $\emptyset$ & $1$ \\
%  $\{a\}$ & $0$
%\end{tabular}
&
\only<2-4>{\color{light}}%
\begin{tabular}[b]{cc}
  & $k_{b, \omega}$ \\
\cline{2-2}
  & $1$
\end{tabular}
%\begin{tabular}[b]{c|c}
%  $\omega$ & $k_{b, \omega}$ \\
%\hline
%  $\emptyset$ & $1$
%\end{tabular}
\\
\only<2-4>{$\underbrace{\text{\hspace{3cm}}}_{\text{Interaction Graph}}$}%
&
\multicolumn{3}{r}{\only<5->{$\underbrace{\text{\hspace{6.5cm}}}_\text{Parametrization}$}}
\end{tabular}

\bigskip
% Historical model...
\only<1>{

\bigskip
Proposed by René Thomas in 1973, several extensions since then
\medskip
\begin{liste}
  \item \tval{Historical bio-informatics model} for studying genes interactions
  \item Widely used and well-adapted to represent dynamic gene systems
\end{liste}
}

% Interaction Graph
\only<2-4>{
\tval{Interaction Graph}: structure of the system (genes \& interactions)

\pause[3]
\medskip
\begin{liste}
  \item \tval{Nodes}: genes
  \begin{fleches}
    \item Name \qex{$a$, $b$, $z$}
    \item Possible values (levels of expression) \qex{$0..1$, $0..2$}
  \end{fleches}
\pause[4]
  \item \tval{Edges}: interactions
  \begin{fleches}
    \item Threshold \qex{$1$}
    \item Type (activation or inhibition) \qex{$+$ / $-$}
  \end{fleches}
\end{liste}
}

% Parametrization
\only<5->{
\medskip
\tval{Parametrization}: strength of the influences (cooperations)

\medskip
\begin{liste}
  \item Maps of tendencies for each gene
  \begin{fleches}
    \item To any \tval{influences of predecessors} \qex{$\omega$}
    \item Corresponds a \tval{parameter} \qex{$k_{x,\omega}$}
  \end{fleches}
\pause[6]
\medskip
  \item \ex{“$k_{z, \{a^+, b^-\}} = 1$”} \quad means: \qex{“$z$ tends to $2$ when activated by $a$ and inhibited by $b$”}
%  \item \ex{“$k_{z, \{a\}} = [2;2]$”} \quad means: \qex{“$z$ tends to \only<7>{$[2;2]$}\only<8->{$2$} when $a \only<7>{\geq}\only<8->{=} 1$ and $b \only<7>{< 1}\only<8->{= 0}$”}
\end{liste}
%\pause[10]
}\end{frame}



\begin{frame}
  \frametitle{Biological Regulatory Network (Thomas' modeling)}
  \framesubtitle{\tcite{\crcbmfma}}

\begin{tabular}{cccc}

\begin{tikzpicture}[grn]
\path[use as bounding box] (-0.7,-0.3) rectangle (2.5,2);
\node[inner sep=0] (z) at (2,0.75) {z};
\node[inner sep=0] (a) at (0,1.5) {a};
\node[inner sep=0] (b) at (0,0) {b};
\path
  node[elabel, below=-1em of a] {$0..1$}
  node[elabel, below=-1em of b] {$0..1$}
  node[elabel, below=-1em of z] {$0..2$};
\path
  (a) edge[inh,loop left=10] node[elabel, left] {$1-$} (a)
  (a) edge[act] node[elabel, above=-2pt] {$1+$} (z)
  (b) edge[inh] node[elabel, below=-2pt] {$1-$} (z);
\end{tikzpicture}
&
\begin{tabular}[b]{c|c|c}
  \multicolumn{2}{c|}{$\omega$} & \multirow{2}{*}{$k_{z, \omega}$} \\
\cline{1-2}
  $a$ & $b$ & \\
\hline
  $-$ & $+$ & $1$ \\
  $-$ & $-$ & $0$ \\
  $+$ & $+$ & $2$ \\
  $+$ & $-$ & $1$
\end{tabular}
~~~~&
\begin{tabular}[b]{c|c}
  $\omega$ & \multirow{2}{*}{$k_{a, \omega}$} \\
\cline{1-1}
  $a$ & \\
\hline
  $+$ & $1$ \\
  $-$ & $0$
\end{tabular}
&
\begin{tabular}[b]{cc}
   & $k_{b, \omega}$ \\
\cline{2-2}
   & $1$
\end{tabular}
\\
\multicolumn{4}{r}{$\underbrace{\text{\hspace{10.1cm}}}_{\text{Biological Regulatory Network}}$}
\end{tabular}

\bigskip

\begin{itemize}
  \item[\f] All needed information to run the model or study its dynamics:
  \begin{itemize}\normalsize
    \item Build the State Graph
    \item Find reachability properties, fixed points, attractors
    \item Other properties...
  \end{itemize}
\end{itemize}

\begin{itemize}
  \item[\f] \tval{Strengths}: well adapted for the study of biological systems
  \item[\f] \tval{Drawbacks}: inherent complexity; needs the full\\
    \quad \quad specification of cooperations
\end{itemize}
\end{frame}


\section{Translating a Process Hitting into a BRN}
%\input{parts/inf.tex}
\subsection{Interaction Graph Inference}
%\input{parts/inf_ig.tex}
\subsection{Parametrization Inference}
%\input{parts/inf_param.tex}
\subsection{Implementation}
% Pistes d'implémentation

\begin{frame}[c]
  \frametitle{ASP Implementation}

\tval{ASP}: Declarative programming

\quad Rule: \ex{$head \leftarrow \only<1>{body}\only<2->{A_1, ..., A_n, \neg A_{n+1}, ..., \neg A_m}.$}

\quad Fact: \ex{$head.$}

\quad Constraint: \ex{$\leftarrow body.$}

\quad Aggregate: \ex{$lower~\{~atoms~\}~upper \leftarrow body.$}

\pause[3]
\bigskip
Representation of PH / BRNs:

\quad Gene: \ex{$component(a, n).$}

\quad Action: \ex{$action(a, i, b, j, k).$}

\quad Cooperation: \ex{$cooperation(c, a, i, j).$}

\quad Useful rules: \ex{$component\_levels(X, 0..M) \leftarrow component(X, M).$}
\end{frame}



% Pistes d'implémentation

\begin{frame}[c]
  \frametitle{ASP Implementation}

\tval{ASP}: Declarative programming

\quad Rule: \texttt{\ex{head :- \only<1>{body}\only<2->{A$_1$, ..., A$_n$, $\neg$A$_{n+1}$, ..., $\neg$A$_m$}.}}

\quad Fact: \texttt{\ex{head.}}

\quad Constraint: \texttt{\ex{:- body.}}

\pause[3]
\bigskip
Example:

\texttt{\ex{%
~~~node(a). node(b). node(c).\\
~~~edge(a, b). edge(b, c). edge(a, c).\\
~~~edge(X,Y) :- edge(Y,X).
}}

\bigskip
Solving: find the biggest set of atoms satisfying the problem

\texttt{\console{%
~~~node(a) node(c) node(b)\\
~~~edge(a, b) edge(b, c) edge(a, c)\\
~~~edge(b, a) edge(c, b) edge(c, a)\\
}}

\end{frame}



\begin{frame}[c]
  \frametitle{ASP Implementation}

\tval{Cardinalities}: \quad
\texttt{\ex{%
\textit{min} \{ atom : enum \} \textit{max} :- body.
}}

\pause
\bigskip

\f Enumerate of all combinations:

\texttt{\ex{%
~~~color(red;green;blue).\\
~~~1 {attrib(X,C) : color(C)} 1 :- node(X).
}}

\texttt{\console{%
Answer~1:~attrib(b,red) attrib(c,red) attrib(a,red)\\
Answer~2:~attrib(b,red) attrib(c,red) attrib(a,blue)\\
Answer~3:~attrib(b,red) attrib(c,green) attrib(a,blue)\\
~~~$\vdots$\\
SATISFIABLE: 27 models
}}

\pause
\bigskip

\f Then reduce the number of answers with constraints:

\texttt{\ex{%
~~~:- attrib(X,C), attrib(Y,C), edge(X,Y).
}}

\texttt{\console{%
Answer~1:~attrib(b,green) attrib(c,blue) attrib(a,red)\\
Answer~2:~attrib(b,green) attrib(c,red) attrib(a,blue)\\
Answer~3:~attrib(b,blue) attrib(c,green) attrib(a,red)\\
~~~$\vdots$\\
SATISFIABLE: 6 models
}}

\end{frame}


\begin{frame}[c]
  \frametitle{Enumerating admissible Parametrizations}
  \framesubtitle{Implementation}

PH / GRN definitions:

\texttt{\ex{%
~~~component(a, 2).\\
~~~component(b, 1).\\
~~~action(b, 1, a, 1, 2).
}}

\bigskip
Useful rules:

\texttt{\ex{%
~~~component\_levels(X, 0..M) :- component(X, M).\\
~~~less\_active(X, P, Q) :-}} $K_{\texttt{X},\texttt{P}}$\text{ has less activators than }$K_{\texttt{X},\texttt{Q}}$\texttt{\ex{\\
~~~param\_inf(X, P, Q) :-}} $K_{\texttt{X},\texttt{P}} \preccurlyeq K_{\texttt{X},\texttt{Q}}$

\pause
\bigskip
Parameters enumeration with cardinalities:

\texttt{\ex{%
~~~1 \{ param(X, P, I) : component\_levels(X, I) \} :-\\
~~~~~param\_label(X, P).
}}

[\texttt{X}: component, \texttt{P}: parameter label, \texttt{I}: parameter value]

\pause
\bigskip
Parametrizations filtering with constraints:

\texttt{\ex{%
~~~:- less\_active(X, P, Q), not param\_inf(X, P, Q).
}}

[\texttt{X}: component, \texttt{P}, \texttt{Q}: parameter labels]

\end{frame}


\section{Summary \& Conclusion}
% Conclusion

\begin{frame}[c]
  \frametitle{Summary}

\begin{enumerate}[1.]
  \item Inference of the \tval{complete Interaction Graph}
  \item Inference of the \tval{possibly partial Parametrization}
  \item Enumerate all full \& \tval{admissible Parametrizations}
\end{enumerate}
\quad\quad\f Exhaustive approaches

\pause
\bigskip
\begin{flushright}
\Large
\textcolor{couleurtheme}{Conclusion}\hspace*{2.7em}
\end{flushright}

\medskip
Existing translation: René Thomas $\rightsquigarrow$ Process Hitting

\smallskip
New translation: Process Hitting $\rightsquigarrow$ René Thomas

\smallskip
\begin{fleches}
  \item New \tval{formal link} between the two models
  \item More \tval{visibility} to the Process Hitting
\end{fleches}
\end{frame}



\section[x]{Acknowledgments}

\begin{frame}[c]
  \frametitle{Joint work}

\tval{Inoue Laboratory}: National Institute of Informatics / Sokendai / Tokyo (Japan)

\smallskip
\tval{MeForBio}: IRCCyN / École Centrale de Nantes / Nantes (France)

\smallskip
\tval{Bioinfo}: LRI / Université Paris-Sud / Orsay (France)

\bigskip\bigskip\footnotesize
\begin{tabular}{cc}
  $\left.\text{\begin{tabular}{c}
    \includegraphics[height=1.5cm]{figs/Inoue-sensei.jpg} \\ \tval{Katsumi INOUE} \\ Professor \& team leader
  \end{tabular}}\right\}\text{\tval{Inoue Laboratory}}$
  &
  $\left.\text{\begin{tabular}{c}
    \includegraphics[height=1.5cm]{figs/Loic.jpg} \\ \tval{Loïc PAULEVÉ} \\ CNRS Researcher
  \end{tabular}}\right\}\text{\tval{Bioinfo}}$
  \\ & \\ & \\
  \multicolumn{2}{l}{$\left.\text{\begin{tabular}{ccc}
      \includegraphics[height=1.5cm]{figs/Olivier.jpg}
    & \includegraphics[height=1.5cm]{figs/Morgan.jpg}
    & \includegraphics[height=1.5cm]{figs/Moi.jpg} \\
      \tval{Olivier ROUX} & \tval{Morgan MAGNIN} & \tval{Maxime FOLSCHETTE} \\
      Professor \& team leader & Associate professor & 2\textsuperscript{nd} year PhD student
  \end{tabular}}\right\}\text{\tval{MeForBio}}$}
\end{tabular}
\end{frame}

\appendix
\section[x]{Bibliography}
% Bibliographie

\begin{frame}[c]
  \frametitle{Bibliography}

\footnotesize
\setlength{\parindent}{-1em}
\setlength{\parskip}{0.5em}
~

\vfill

\tcite{PMR10-TCSB} Loïc Paulevé, Morgan Magnin, Olivier Roux. \ex{Refining dynamics of gene regulatory networks in a stochastic $\pi$-calculus framework}. In Corrado Priami, Ralph-Johan Back, Ion Petre, and Erik de Vink, editors: \textit{Transactions on Computational Systems Biology XIII}, volume 6575 of Lecture Notes in Computer Science, 171-191. Springer Berlin/Heidelberg, 2011.

\tcite{PMR12-MSCS} Loïc Paulevé, Morgan Magnin, Olivier Roux. \ex{Static analysis of biological regulatory networks dynamics using abstract interpretation}. \textit{Mathematical Structures in Computer Science}, 2012.

%\tcite{RCB08} Adrien Richard, Jean-Paul Comet, Gilles Bernot. \ex{R. Thomas' logical method}, 2008. Invited at \textit{Tutorials on modelling methods and tools: Modelling a genetic switch and Metabolic Networks}, Spring School on Modelling Complex Biological Systems in the Context of Genomics.

\tcite{RCB06} Adrien Richard, Jean-Paul Comet, Gilles Bernot. \textit{Modern Formal Methods and App.}, chapter \ex{Formal Methods for Modeling Biological Regulatory Networks}, pages 83--122. 2006.


\tcite{CMSB12} Maxime Folschette, Loïc Paulevé, Katsumi Inoue, Morgan Magnin, Olivier Roux. \ex{Concretizing the Process Hitting into Biological Regulatory Networks}. In David Gilbert and Monika Heiner, editors, \textit{Computational Methods in Systems Biology X}, Lecture Notes in Computer Science, pages 166–186. Springer Berlin
Heidelberg, 2012.

%\tcite{Paulevé11} Loïc Paulevé. PhD thesis: \ex{\textit{Modélisation, Simulation et Vérification des Grands Réseaux de Régulation Biologique}}, October 2011, Nantes, France

%\tcite{PMR10-TSE} Loïc Paulevé, Morgan Magnin, and Olivier Roux. \textit{Tuning Temporal Features within the Stochastic $\pi$-Calculus}. IEEE Transactions on Software Engineering, 37(6):858-871, 2011.

%\tcite{PR10-CRAS} Loïc Paulevé and Adrien Richard. \textit{Topological Fixed Points in Boolean Networks}. Comptes Rendus de l'Académie des Sciences - Series I - Mathematics, 348(15-16):825 - 828, 2010.

\vfill
\Large
\begin{flushright}
  \tval{Thank you}\hspace{1cm}~
\end{flushright}
\vfill

~

\end{frame}

\section[Annex: Graphs of local causality]{Graphs of local causality}
%% Exemples de structures abstraites (graphes de causalité locale)

\begin{frame}
  \frametitle{Under-approximation}

\def \tu {2}
\def \tub {3}
\def \tuf {4}

\begin{columns}
\begin{column}{0.48\textwidth}
\begin{center}
\scalebox{0.55}{
\begin{tikzpicture}
\exatt
\TState{-\tu}{a_1,b_1,c_1,d_0}
\TState{\tub-}{a_0,b_1,c_0,d_0}
\node[process,very thick] (d_2) at (d_2.center) {?};
\end{tikzpicture}
}
\end{center}

\end{column}
\begin{column}{0.52\textwidth}

\vspace{1.5em}
\tval{Sufficient condition}:

\smallskip
\begin{itemize}
  \item no cycle
  \item \only<-\tu>{each objective has a solution} \only<\tub->{\sout{each objective has a solution}}
\end{itemize}
\begin{center}
  \only<\tu>{\Large\textcolor{darkgreen}{$R$ is \textbf{true}}} \only<\tuf>{\Large\textcolor{darkyellow}{\textbf{Inconclusive}}}
\end{center}

\end{column}
\end{columns}

\only<-\tu>{
\sauyes
}

\only<\tub->{
\sauinconc
}

\end{frame}



\begin{frame}
  \frametitle{Over-approximation}

\def \to {4}
\def \tob {5}
\def \tof {6}
\def \tokp {7}

\begin{columns}
\begin{column}{0.48\textwidth}
\begin{center}
\scalebox{0.55}{
\begin{tikzpicture}
\exatt
\TState{-\to}{a_1,b_0,c_0,d_1}
\TState{\tob-}{a_1,b_1,c_1,d_0}
\node[process,very thick] (d_2) at (d_2.center) {?};
\end{tikzpicture}
}
\end{center}
\bigskip

\end{column}
\begin{column}{0.52\textwidth}

\tval{Necessary condition}:

\smallskip
\only<2->{
\only<3-\to>{\sout{There exists a traversal}}\only<2,\tob->{There exists a traversal}
with no cycle

\smallskip
\begin{itemize}
  \item \only<3-\to>{\sout{objective $\rightarrow$ follow one solution}}\only<1-2,\tob->{objective $\rightarrow$ follow one solution}
  \item solution $\rightarrow$ follow all processes
  \item process $\rightarrow$ follow all objectives
\end{itemize}
\begin{center}
  \only<\to>{\Large\textcolor{red}{$R$ is \textbf{false}}}\only<\tof->{\Large\textcolor{darkyellow}{\textbf{Inconclusive}}}
\end{center}
}

\end{column}
\end{columns}

%\bigskip

\only<1-\to>{
\saono
}

\only<\tob->{
\saoinconc
}

\end{frame}

%\input{parts/annex_ph.tex}

\end{document}
