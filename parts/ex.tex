% Exemples

%%% Exemple pour la définition du Process Hitting %%%
\def \exphdef {
\path[use as bounding box] (-0.5,-0.5) rectangle (6.5,4.5);

\TSort{(0,3)}{a}{2}{l}
\TSort{(0,0)}{b}{2}{l}
\TSort{(6,1)}{z}{3}{r}

\THit{a_1}{}{z_1}{.west}{z_2}
\THit{b_1}{}{z_0}{.west}{z_1}
\THit{a_0}{out=250,in=200,selfhit}{a_0}{.west}{a_1}

\path[bounce,bend left]
\TBounce{z_0}{}{z_1}{.south}
\TBounce{z_1}{}{z_2}{.south}
\TBounce{a_0}{}{a_1}{.south}
;
}



%%% Exemple pour la coopération %%%
\def \exphcoop {
\path[use as bounding box] (-0.5,-0.5) rectangle (6.5,4.5);

% Actions de màj grisées
\only<6->{
\THit{a_1}{ulhit,color=lightgray}{ab_0}{.west}{ab_2}
\THit{a_1}{ulhit,color=lightgray}{ab_1}{.west}{ab_3}
\path[bounce,bend left,pulhit] \TBounce{ab_0}{bulhit}{ab_2}{.south} \TBounce{ab_1}{bulhit}{ab_3}{.south} ;
}

\only<7->{
\THit{a_0}{ulhit}{ab_2}{.west}{ab_0}
\THit{a_0}{ulhit}{ab_3}{.west}{ab_1}
\path[bounce,bend right,pulhit] \TBounce{ab_2}{bulhit}{ab_0}{.north} \TBounce{ab_3}{bulhit}{ab_1}{.north} ;
}

\only<8->{
\THit{b_0}{ulhit}{ab_3}{.west}{ab_2}
\THit{b_0}{ulhit}{ab_1}{.west}{ab_0}
\THit{b_1}{ulhit}{ab_0}{.west}{ab_1}
\THit{b_1}{ulhit}{ab_2}{.west}{ab_3}
\path[bounce,bend right,pulhit] \TBounce{ab_1}{bulhit}{ab_0}{.north} \TBounce{ab_3}{bulhit}{ab_2}{.north} ;
\path[bounce,bend left,pulhit] \TBounce{ab_0}{bulhit}{ab_1}{.south} \TBounce{ab_2}{bulhit}{ab_3}{.south} ;
}

% Sortes
\TSort{(0,3)}{a}{2}{l}
\TSort{(0,0)}{b}{2}{l}
\TSort{(6,1)}{z}{3}{r}

% Deux actions disjointes en exemple
\only<2-3>{
\THit{a_1}{}{z_1}{.north west}{z_2}
\path[bounce,bend left]
\TBounce{z_1}{}{z_2}{.south} ;

\THit{b_0}{}{z_1}{.west}{z_2}
\path[bounce,bend left=55]
\TBounce{z_1}{}{z_2}{.south west} ;
}

% Processus d'exemple
\TState{3}{a_1,b_1,z_1}

% Sorte coopérative et arcs
\only<4->{
\TSetTick{ab}{0}{00}
\TSetTick{ab}{1}{01}
\TSetTick{ab}{2}{10}
\TSetTick{ab}{3}{11}
\TSort{(3,0.5)}{ab}{4}{l}
}

% Arcs de màj noirs de la sc
\only<5>{
\THit{a_1}{thick}{ab_0}{.west}{ab_2}
\THit{a_1}{thick}{ab_1}{.west}{ab_3}
\path[bounce,thick,bend left] \TBounce{ab_0}{thick}{ab_2}{.south} \TBounce{ab_1}{thick}{ab_3}{.south} ;
}

\only<6>{
\THit{a_0}{thick}{ab_2}{.west}{ab_0}
\THit{a_0}{thick}{ab_3}{.west}{ab_1}
\path[bounce,thick,bend right] \TBounce{ab_2}{thick}{ab_0}{.north} \TBounce{ab_3}{thick}{ab_1}{.north} ;
}

\only<7>{
\THit{b_0}{thick}{ab_3}{.west}{ab_2}
\THit{b_0}{thick}{ab_1}{.west}{ab_0}
\THit{b_1}{thick}{ab_0}{.west}{ab_1}
\THit{b_1}{thick}{ab_2}{.west}{ab_3}
\path[bounce,thick,bend right] \TBounce{ab_1}{thick}{ab_0}{.north} \TBounce{ab_3}{thick}{ab_2}{.north} ;
\path[bounce,thick,bend left] \TBounce{ab_0}{thick}{ab_1}{.south} \TBounce{ab_2}{thick}{ab_3}{.south} ;
}

% État d'exemple pour màj de la sc
\TState{8-9}{a_1,b_0}
\TState{10}{a_1,b_0,ab_0,ab_1,ab_2,ab_3}
\TState{11}{a_1,b_0,ab_2}
\only<9-11>{
\THit{a_1}{}{ab_0}{.west}{ab_2}
\THit{a_1}{}{ab_1}{.west}{ab_3}
\THit{b_0}{}{ab_3}{.west}{ab_2}
\THit{b_0}{}{ab_1}{.west}{ab_0}
\path[bounce,bend left] \TBounce{ab_0}{}{ab_2}{.south} \TBounce{ab_1}{}{ab_3}{.south} ;
\path[bounce,bend right] \TBounce{ab_1}{}{ab_0}{.north} \TBounce{ab_3}{}{ab_2}{.north} ;
}

% État d'exemple pour action de la sc
\TState{12}{a_1,b_0,z_1,ab_2}
\TState{13-14}{a_1,b_0,z_2,ab_2}

% Arc sortant de la sc
\only<12-14>{
\THit{ab_2}{thick}{z_1}{.west}{z_2}
\path[bounce,bend left,thick] \TBounce{z_1}{thick}{z_2}{.south} ;
}

% Arc sortant de la sc
%\only<15->{
%\THit{ab_2}{}{z_1}{.west}{z_2}
%\path[bounce,bend left] \TBounce{z_1}{}{z_2}{.south} ;
%}

}



%%% Exemple pour l'inférence %%%
\def \exphinf {
% Sortes
\TSort{(0,3)}{a}{2}{l}
\TSort{(0,0)}{b}{2}{l}
\TSort{(6,0)}{z}{3}{r}

% Sorte coopérative et arcs
\TSetTick{ab}{0}{00}
\TSetTick{ab}{1}{01}
\TSetTick{ab}{2}{10}
\TSetTick{ab}{3}{11}
\TSort{(3,0)}{ab}{4}{l}

% Actions de màj grisées
\THit{a_1}{ulhit}{ab_0}{.west}{ab_2}
\THit{a_1}{ulhit}{ab_1}{.west}{ab_3}
\path[bounce,bend left,pulhit] \TBounce{ab_0}{bulhit}{ab_2}{.south} \TBounce{ab_1}{bulhit}{ab_3}{.south};

\THit{a_0}{ulhit}{ab_2}{.west}{ab_0}
\THit{a_0}{ulhit}{ab_3}{.west}{ab_1}
\path[bounce,bend right,pulhit] \TBounce{ab_2}{bulhit}{ab_0}{.north} \TBounce{ab_3}{bulhit}{ab_1}{.north};

\THit{b_0}{ulhit}{ab_3}{.west}{ab_2}
\THit{b_0}{ulhit}{ab_1}{.west}{ab_0}
\THit{b_1}{ulhit}{ab_0}{.west}{ab_1}
\THit{b_1}{ulhit}{ab_2}{.west}{ab_3}
\path[bounce,bend right,pulhit] \TBounce{ab_1}{bulhit}{ab_0}{.north} \TBounce{ab_3}{bulhit}{ab_2}{.north};
\path[bounce,bend left,pulhit] \TBounce{ab_0}{bulhit}{ab_1}{.south} \TBounce{ab_2}{bulhit}{ab_3}{.south};

% Arcs sortant de la sc
\THit{ab_2}{ulhit}{z_1}{.north west}{z_2}
\THit{ab_2}{ulhit}{z_0}{.west}{z_1}
\path[bounce,bend left,pulhit] \TBounce{z_1}{bulhit}{z_2}{.south} \TBounce{z_0}{bulhit}{z_1}{.south};

\THit{ab_3}{ulhit}{z_2}{.west}{z_1}
\THit{ab_3}{ulhit}{z_0}{.west}{z_1}
\THit{ab_1}{ulhit}{z_2}{.west}{z_1}
\THit{ab_1}{ulhit}{z_0}{.west}{z_1}
\path[bounce,bend left,pulhit] \TBounce{z_2}{bulhit,bend right}{z_1}{.north};

\THit{ab_0}{ulhit}{z_2}{.west}{z_1}
\THit{ab_0}{ulhit}{z_1}{.south west}{z_0}
\path[bounce,bend right,pulhit] \TBounce{z_2}{bulhit}{z_1}{.north} \TBounce{z_1}{bulhit}{z_0}{.north};

}



%%% Exemple pour l'inférence (sans arcs grisés) %%%
\def \exphinfblack {
% Sortes
\TSort{(0,3)}{a}{2}{l}
\TSort{(0,0)}{b}{2}{l}
\TSort{(6,0)}{z}{3}{r}

% Sorte coopérative et arcs
\TSetTick{ab}{0}{00}
\TSetTick{ab}{1}{01}
\TSetTick{ab}{2}{10}
\TSetTick{ab}{3}{11}
\TSort{(3,0)}{ab}{4}{l}

% Actions de màj grisées
\THit{a_1}{}{ab_0}{.west}{ab_2}
\THit{a_1}{}{ab_1}{.west}{ab_3}
\path[bounce,bend left] \TBounce{ab_0}{}{ab_2}{.south} \TBounce{ab_1}{}{ab_3}{.south};

\THit{a_0}{}{ab_2}{.west}{ab_0}
\THit{a_0}{}{ab_3}{.west}{ab_1}
\path[bounce,bend right] \TBounce{ab_2}{}{ab_0}{.north} \TBounce{ab_3}{}{ab_1}{.north};

\THit{b_0}{}{ab_3}{.west}{ab_2}
\THit{b_0}{}{ab_1}{.west}{ab_0}
\THit{b_1}{}{ab_0}{.west}{ab_1}
\THit{b_1}{}{ab_2}{.west}{ab_3}
\path[bounce,bend right] \TBounce{ab_1}{}{ab_0}{.north} \TBounce{ab_3}{}{ab_2}{.north};
\path[bounce,bend left] \TBounce{ab_0}{}{ab_1}{.south} \TBounce{ab_2}{}{ab_3}{.south};

% Arcs sortant de la sc
\THit{ab_2}{}{z_1}{.north west}{z_2}
\THit{ab_2}{}{z_0}{.west}{z_1}
\path[bounce,bend left] \TBounce{z_1}{}{z_2}{.south} \TBounce{z_0}{}{z_1}{.south};

\THit{ab_3}{}{z_2}{.west}{z_1}
\THit{ab_3}{}{z_0}{.west}{z_1}
\THit{ab_1}{}{z_2}{.west}{z_1}
\THit{ab_1}{}{z_0}{.west}{z_1}
\path[bounce,bend left] \TBounce{z_2}{,bend right}{z_1}{.north};

\THit{ab_0}{}{z_2}{.west}{z_1}
\THit{ab_0}{}{z_1}{.south west}{z_0}
\path[bounce,bend right] \TBounce{z_2}{}{z_1}{.north} \TBounce{z_1}{}{z_0}{.north};

}



%%% Exemple 2 pour l'inférence (projections) %%%
\def \exphinfproj {
% Sortes
\TSort{(0,3)}{a}{2}{l}
\TSort{(0,0)}{b}{2}{l}
\TSort{(6,1)}{z}{2}{r}

% Sorte coopérative et arcs
\TSetTick{ab}{0}{00}
\TSetTick{ab}{1}{01}
\TSetTick{ab}{2}{10}
\TSetTick{ab}{3}{11}
\TSort{(3,0)}{ab}{4}{l}

% Actions de màj grisées
\THit{a_1}{ulhit}{ab_0}{.west}{ab_2}
\THit{a_1}{ulhit}{ab_1}{.west}{ab_3}
\path[bounce,bend left,pulhit] \TBounce{ab_0}{bulhit}{ab_2}{.south} \TBounce{ab_1}{bulhit}{ab_3}{.south} ;

\THit{a_0}{ulhit}{ab_2}{.west}{ab_0}
\THit{a_0}{ulhit}{ab_3}{.west}{ab_1}
\path[bounce,bend right,pulhit] \TBounce{ab_2}{bulhit}{ab_0}{.north} \TBounce{ab_3}{bulhit}{ab_1}{.north} ;

\THit{b_0}{ulhit}{ab_3}{.west}{ab_2}
\THit{b_0}{ulhit}{ab_1}{.west}{ab_0}
\THit{b_1}{ulhit}{ab_0}{.west}{ab_1}
\THit{b_1}{ulhit}{ab_2}{.west}{ab_3}
\path[bounce,bend right,pulhit] \TBounce{ab_1}{bulhit}{ab_0}{.north} \TBounce{ab_3}{bulhit}{ab_2}{.north} ;
\path[bounce,bend left,pulhit] \TBounce{ab_0}{bulhit}{ab_1}{.south} \TBounce{ab_2}{bulhit}{ab_3}{.south} ;

% Arcs sortant de la sc
\THit{ab_3}{ulhit}{z_0}{.west}{z_1}
\path[bounce,bend left,pulhit] \TBounce{z_0}{bulhit}{z_1}{.south} ;

\THit{ab_0}{ulhit}{z_1}{.west}{z_0}
\path[bounce,bend right,pulhit]\TBounce{z_1}{bulhit}{z_0}{.north} ;
}



%%% Exemple sans sorte coopérative pour l'inférence %%%
\def \exphinfprojssc {
% Sortes
\TSort{(0,3)}{a}{2}{l}
\TSort{(0,0)}{b}{2}{l}
\TSort{(6,0)}{z}{3}{r}

\THit{a_1}{ulhit}{z_0}{.west}{z_1}
\THit{a_1}{ulhit}{z_1}{.north west}{z_2}
\THit{a_0}{ulhit}{z_1}{.south west}{z_0}
\THit{a_0}{ulhit}{z_2}{.west}{z_1}
\path[bounce,bend left,pulhit] \TBounce{z_0}{bulhit}{z_1}{.south} \TBounce{z_1}{bulhit}{z_2}{.south}
  \TBounce{z_1}{bulhit,bend right}{z_0}{.north} \TBounce{z_2}{bulhit,bend right}{z_1}{.north} ;

\THit{b_0}{ulhit}{z_0}{.west}{z_1}
\THit{b_0}{ulhit}{z_1}{.north west}{z_2}
\THit{b_1}{ulhit}{z_1}{.south west}{z_0}
\THit{b_1}{ulhit}{z_2}{.west}{z_1}
%\path[bounce,bend left,pulhit] \TBounce{z_0}{bulhit}{z_1}{.south} \TBounce{z_1}{bulhit}{z_2}{.south}
%  \TBounce{z_1}{bulhit,bend right}{z_0}{.north} \TBounce{z_2}{bulhit,bend right}{z_1}{.north} ;
}



%%% Exemple atteignabilité
\def \exatt {
\path[use as bounding box] (-1,-3) rectangle (7,2.7);
\TSort{(0,0)}{a}{2}{l}
\TSort{(3,0)}{b}{3}{l}
\TSort{(6,0)}{d}{3}{r}
\TSort{(2,-2)}{c}{2}{b}

\THit{a_0}{}{c_0}{.north}{c_1}
\THit{a_1}{}{b_1}{.west}{b_0}
\THit{c_1}{bend left=20pt}{b_0}{.west}{b_1}
\THit{b_1.south west}{->}{a_0}{.east}{a_1}
\THit{b_0}{}{d_0}{.west}{d_1}
\THit{b_1}{}{d_1}{.west}{d_2}
\THit{d_1}{}{b_0}{.north east}{b_2}
\THit{c_1}{bend right=80pt,distance=80pt}{d_1}{.east}{d_0}
\THit{b_2}{distance=120pt,out=30,in=40}{d_0}{.east}{d_2}

\path[bounce,bend left]
\TBounce{d_0}{}{d_1}{.south}
\TBounce{d_1}{}{d_2}{.south}
\TBounce{c_0}{}{c_1}{.west}
\TBounce{b_0}{}{b_1}{.south}
\TBounce{d_1}{}{d_0}{.north}
;
\path[bounce,bend right]
\TBounce{a_0}{}{a_1}{.south}
\TBounce{b_0}{}{b_2}{.south}
\TBounce{b_1}{}{b_0}{.north}
\TBounce{d_0}{bend right=50pt,distance=40pt}{d_2}{.south}
;
}



%%% Figure de présentation de l'analyse d'atteignabilité
\def \figsa {
\begin{tikzpicture}
\path[use as bounding box] (-5,-3.5) rectangle (5,3.5);
\definecolor{r2}{RGB}{238,10,38}

\path<2->[shading=1, inner color=r2, outer color=white] (3.5,-2.8) -- (4.4,3.2) -- (0,3) -- (-4.5,1.4) -- (-2.5,-2.5) -- (0,-3.6) -- (2.8,-2.8);
%\path<2->[shading, inner color=r2, outer color=white, border color=white] (2.8,-2.8) -- (4.5,4.5) -- (0,3.9) -- (-4.5,1.8) -- (-5,-3) -- (0,-3.2) -- (2.8,-2.8);
\draw<2->[thick,fill=white] (2.5,-2.1) -- (3,2.5) -- (-2.7,1.3) -- (-2,-2) -- (2.5,-2.1);
\draw<6->[thick,fill=lightyellow] (2.5,-2.1) -- (3,2.5) -- (-2.7,1.3) -- (-2,-2) -- (2.5,-2.1);

\node<2->[text width=3.5cm, color=red] (s1) at (-5,2) {Over-Approximation};
\path<2->[->,very thick,color=red] (s1.south) edge (-3.5,1.2);
%\node<2->[text width=3cm,color=black] (i1) at (3.7,.2) {$\Rightarrow$};
\node<2->[text width=3cm,color=black] (q) at (4.5,.2) {$\neg Q$};

%\draw<4->[thick, fill=green] (.5,-.8) -- (1,0) -- (.3,1) -- (-1,.5) -- (-.5,-.5) -- (.5,-.8);
\draw<4->[thick, shading=1, top color=darkgreen, bottom color=green] (.5,-.8) -- (1,0) -- (.3,1) -- (-1,.5) -- (-.5,-.5) -- (.5,-.8);
\node<4->[text width=3.5cm,color=darkgreen] (s2) at (5.2,-1.5) {Under-Approximation};
\node<4->[text width=3cm,color=black] (p) at (1.8,.2) {$P$};
%\node<4->[text width=3cm,color=black] (i1) at (2.25,.2) {$\Rightarrow$};

% reaching set
\node[text width=3cm,color=darkcyan] (s) at (1.8,1.7) {Exact solution};
\node<1->[text width=3cm,color=darkcyan] (s0) at (0,0) {};
\draw[color=darkcyan, thick] (0,0) ellipse (2 and 1.5);
%\path<1>[draw=white] (2.8,-2.8) -- (4.5,4.5) -- (0,3.9) -- (-4.5,1.8) -- (-5,-3) -- (-2.5,-3.5) -- (0,-3.2) -- (2.8,-2.8);
\node[text width=3cm,color=black] (r) at (2.8,.2) {$R$};

\path<4->[->,very thick,color=darkgreen] (s2) edge (.6,-.4);

\tikzstyle{point}=[circle,draw=blue,fill=blue,minimum size=5pt,inner sep=0pt]

%\only<5->{
\only<3->{
\node[point] at (-2.4,-2) {};
\node[point] at (-2,2) {};
}
\only<5->{
\node[point] at (0,0) {};
}
\only<7->{
\node[point] at (-.5,-1.1) {};
\node[point] at (2.5,1) {};
}
%}

\end{tikzpicture}
}




%%% Exemple pour points fixes
\def \exdefb {
\path[use as bounding box] (0,-1) rectangle (4,4);

\TSort{(0,0)}{z}{3}{l}
\TSort{(2,4)}{b}{2}{t}
\TSort{(4,1)}{a}{2}{r}
}

%%% Frappes pour l'exemple des points fixes
\def \exdefbfrappes {
\THit{b_0}{}{z_1}{.east}{z_2}
\THit{b_1}{}{z_0}{.north east}{z_2}
\THit{a_0}{}{b_1}{.south}{b_0}
\THit{a_1}{out=60,in=0,selfhit}{a_1}{.east}{a_0}

\path[bounce,bend right]
\TBounce{z_1}{}{z_2}{.south}
\TBounce{z_0}{bend right=50}{z_2}{.south east}
;
\path[bounce,bend left]
\TBounce{a_1}{}{a_0}{.north}
\TBounce{b_1}{}{b_0}{.south}
;

\onslide<1-3> { \THit{z_0}{}{a_0}{.west}{a_1} }
\onslide<4> { \THit{z_0}{very thick}{a_0}{.west}{a_1} }

\only<1-3>{
\path[bounce,bend left]
\TBounce{a_0}{}{a_1}{.south}
;}

\only<4>{
\path[bounce,bend left,very thick]
\TBounce{a_0}{very thick}{a_1}{.south}
;}
}

%%% Non-frappes pour l'exemple des points fixes
\def \exdefbsf {
\path[use as bounding box] (0,-1) rectangle (4,4);
\node[process,draw=red,thick] (a_1) at (a_1.center) {};

\path (z_0) edge (b_0) (b_0) edge (a_0);
\path (z_2) edge (b_0) edge (a_0);
\path (z_2) edge (b_1);
\path (z_1) edge (b_1) edge (a_0);
%\path<-3> (z_0) edge (a_0);

\path<3->[very thick] (z_2) edge (b_0) edge (a_0) (b_0) edge (a_0);
\path<4->[very thick] (z_0) edge (b_0);
\path<4>[dashed, very thick] (z_0) edge (a_0);
\path<5->[very thick] (z_0) edge (a_0);

\TState{3-}{z_2,b_0,a_0}
\TState{4-}{z_0,b_0,a_0}
}



%%% Exemples de graphes d'atteignabilité

% Structure abstraite / Sous-approximation / Ok
\def \sauyes {%
\begin{tikzpicture}[aS,node distance=1.1cm,shorthandon]
\path[use as bounding box] (-0.5,-2.1) rectangle (10.25,2.2);

\node[Aobj] (d02) {$\PHobjectif{d_0}{d_2}$};
\node[Aproc,above of=d02] (d2) {$d_2$};

\node[Asol,right of=d02] (d02s2) {};
\node[Aproc,above right of=d02s2] (b0) {$b_0$};
\node[Aobj,right of=b0] (b10) {$\PHobjectif{b_1}{b_0}$};
\node[Asol,right of=b10] (b10s) {};
\node[Aproc,right of=b10s] (a1) {$a_1$};
\node[Aobj,right of=a1] (a11) {$\PHobjectif{a_1}{a_1}$};
\node[Asol,right of=a11] (a11s) {};

\node[Aobj,above of=b10,yshift=-0.5cm] (b00)
{$\PHobjectif{b_0}{b_0}$};
\node[Asol,right of=b00] (b00s) {};

\node[Aproc, below of=b0] (b1) {$b_1$};
\node[Aobj,right of=b1] (b11) {$\PHobjectif{b_1}{b_1}$};
\node[Asol,right of=b11] (b11s) {};
\node[Aobj,below of=b11] (b01) {$\PHobjectif{b_0}{b_1}$};
\node[Asol,right of=b01] (b01s) {};
\node[Aproc,right of=b01s] (c1) {$c_1$};
\node[Aobj,right of=c1] (c11) {$\PHobjectif{c_1}{c_1}$};
\node[Asol,right of=c11] (c11s) {};

\path
(d02) edge (d02s2) (d02s2) edge (b1) edge (b0)
(a11) edge (a11s)
(b10) edge (b10s) (b10s) edge (a1)
(b11) edge (b11s)
(b0) edge (b10) (b1) edge (b11)
(a1) edge (a11)
(d2) edge (d02)
;
\path
(b0) edge (b00.west) (b00) edge (b00s)
(b1) edge (b01)
(b01) edge (b01s) (b01s) edge (c1)
(c1) edge (c11) (c11) edge (c11s)
;
%\node<\tu>[right of=a11s] {\textbf{\Large\color{darkgreen}Yes}};
\end{tikzpicture}%
}

% Structure abstraite / Sous-approximation / Inconclusif
\def \sauinconc {%
\begin{tikzpicture}[aS,node distance=1.1cm,shorthandon]
\path[use as bounding box] (-0.5,-2.1) rectangle (10.25,2.2);

\node[Aobj] (d02) {$\PHobjectif{d_0}{d_2}$};
\node[Aproc,above of=d02] (d2) {$d_2$};

\node[Asol,right of=d02] (d02s2) {};
\node[Aproc,above right of=d02s2] (b0) {$b_0$};
\node[Aobj,right of=b0] (b10) {$\PHobjectif{b_1}{b_0}$};
\node[Asol,right of=b10] (b10s) {};
\node[Aproc,right of=b10s] (a1) {$a_1$};
\node[Aobj,right of=a1] (a01) {$\PHobjectif{a_0}{a_1}$};
\node[Asol,right of=a01] (a01s) {};

\node[Aproc, below of=b0] (b1) {$b_1$};
\node[Aobj,right of=b1] (b11) {$\PHobjectif{b_1}{b_1}$};
\node[Asol,right of=b11] (b11s) {};
\node[Aobj,below of=b11] (b01) {$\PHobjectif{b_0}{b_1}$};
\node[Asol,right of=b01] (b01s) {};
\node[Aproc,right of=b01s] (c1) {$c_1$};
\node[Aobj,right of=c1] (c01) {$\PHobjectif{c_0}{c_1}$};
\node[Asol,right of=c01] (c01s) {};
\node[Aproc,right of=c01s] (a0) {$a_0$};
\node[Aobj,right of=a0] (a00) {$\PHobjectif{a_0}{a_0}$};
\node[Asol,right of=a00] (a00s) {};

\node[Aobj,above of=b10] (b00) {$\obj{b_0}{b_0}$};
\node[Asol,right of=b00] (b00s) {};
\node[Aobj,above of=a01] (a11) {$\obj{a_1}{a_1}$};
\node[Asol,right of=a11] (a11s) {};
\node[Aobj,above of=c01] (c11) {$\obj{c_1}{c_1}$};
\node[Asol,right of=c11] (c11s) {};
\node[Aobj,above of=a00] (a10) {$\PHobjectif{a_1}{a_0}$};
\node at (a10.east) {\Large\color{red}\textbf{$\bot$}};

\path
  (b10) edge[loop,min distance=5mm] (b10)
 ;
\path
(d02) edge (d02s2) (d02s2) edge (b1) edge (b0)
(a01) edge (a01s) (a01s.south) edge (b1.north east)
(b10) edge (b10s) (b10s) edge (a1)
(b11) edge (b11s)
(a1) edge (a01)
(b0) edge (b10) (b1) edge (b11)
(d2) edge (d02)
;
\path
(b00) edge (b00s)
(b0) edge (b00)
 (b1) edge (b01)
 (b01) edge (b01s) (b01s) edge (c1)
 (c1) edge (c01)
 (c01) edge (c01s) (c01s) edge (a0)
 (a0) edge (a00) (a00) edge (a00s)
;
\path
 (c1) edge (c11) (c11) edge (c11s)
(a0) edge (a10)
(a1) edge (a11)
(a11) edge (a11s)
;

%\node[right of=a01s] {\textbf{\Large\color{darkyellow}Inconc}};

\end{tikzpicture}%
}

% Structure abstraite / Sur-approximation / Non
\def \saono {%
\begin{tikzpicture}[aS,node distance=1.1cm,shorthandon]
\path[use as bounding box] (-0.5,-2.1) rectangle (10.25,1.15);

\node[Aobj] (d12) {$\PHobjectif{d_1}{d_2}$};
\node[Asol,above right of=d12] (d12s1) {};
\node[Aproc, right of=d12s1] (b2) {$b_2$};
\node[Aobj,right of=b2] (b02) {$\PHobjectif{b_0}{b_2}$};
\node[Asol,right of=b02] (b02s) {};
\node[Aproc,right of=b02s] (d1) {$d_1$};
\node[Aobj,right of=d1] (d11) {$\PHobjectif{d_1}{d_1}$};
\node[Asol,right of=d11] (d11s) {};

\node[Asol,below right of=d12] (d12s2) {};
\node[Aproc, right of=d12s2] (b1) {$b_1$};
\node[Aobj,right of=b1] (b01) {$\PHobjectif{b_0}{b_1}$};
\node[Asol,right of=b01] (b01s) {};
\node[Aproc,right of=b01s] (c1) {$c_1$};
\node[Aobj,right of=c1] (c01) {$\PHobjectif{c_0}{c_1}$};
\node[Asol,right of=c01] (c01s) {};
\node[Aproc,right of=c01s] (a0) {$a_0$};
\node[Aobj,right of=a0] (a10) {$\PHobjectif{a_1}{a_0}$};
\node at (a10.east) {\Large\color{red}\textbf{$\bot$}};

\path
(d12) edge (d12s1) edge (d12s2) (d12s1) edge (b2) edge (c1) (d12s2) edge (b1)
(b01) edge (b01s) (b01s) edge (c1)
(b02) edge (b02s) (b02s) edge (d1)
(c01) edge (c01s) (c01s) edge (a0)
(d11) edge (d11s)
(a0) edge (a10)
(b1) edge (b01)
(b2) edge (b02)
(c1) edge (c01)
(d1) edge (d11)
;
%\only<\value{anim1}>{ \node[above right of=c01s] {\textbf{\Large\color{red}No}};}
\end{tikzpicture}%
}

% Structure abstraite / Sur-approximation / Inconclusif
\def \saoinconc {%
\begin{tikzpicture}[aS,node distance=1.1cm,shorthandon]
\path[use as bounding box] (-0.5,-2.1) rectangle (10.25,1.15);

\node[Aobj] (d02) {$\PHobjectif{d_0}{d_2}$};
\node[Asol,above right of=d02] (d02s1) {};

\node[Aproc, right of=d02s1] (b2) {$b_2$};
\node[Aobj,right of=b2] (b12) {$\PHobjectif{b_1}{b_2}$};
\node[Asol,right of=b12] (b12s) {};
\node[Aproc,right of=b12s] (d1) {$d_1$};
\node[Aobj,right of=d1] (d01) {$\PHobjectif{d_0}{d_1}$};
\node[Asol,right of=d01] (d01s) {};

\node[Asol,below right of=d02] (d02s2) {};
%<-3>
\node<-\tof>[Aproc, right of=d02s2] (b0) {$b_0$};
\node<\tokp>[orange, thick, Aproc, right of=d02s2] (b0) {$b_0$};
\node[Aobj,right of=b0] (b10) {$\PHobjectif{b_1}{b_0}$};
\node[Asol,right of=b10] (b10s) {};
%<-3>
\node<-\tof>[Aproc,right of=b10s] (a1) {$a_1$};
\node<\tokp>[orange, thick, Aproc,right of=b10s] (a1) {$a_1$};
\node[Aobj,right of=a1] (a11) {$\PHobjectif{a_1}{a_1}$};
\node[Asol,right of=a11] (a11s) {};

\node[Aproc, below of=b0] (b1) {$b_1$};
\node[Aobj,right of=b1] (b11) {$\PHobjectif{b_1}{b_1}$};
\node[Asol,right of=b11] (b11s) {};

\node<\tokp>[orange, font=\bfseries,below of=a11s] (kp) {Key processes};
\path<\tokp>[orange, thick]
        (kp) edge (a1)
        (kp) edge (b0)
;
\path
(d02) edge (d02s1) edge (d02s2) (d02s1) edge (b2) (d02s2) edge (b1) edge (b0)
(a11) edge (a11s)
(b10) edge (b10s) (b10s) edge (a1)
(b11) edge (b11s)
(b12) edge (b12s) (b12s) edge (d1) edge (a1)
(d01) edge (d01s) (d01s.south) edge (b0)
(a1) edge (a11)
(b0) edge (b10) (b1) edge (b11) (b2) edge (b12)
(d1) edge (d01)
;
%\node[below right of=d01s] {\textbf{\Large\color{yellow}Inconc}};
\end{tikzpicture}%
}