% Définition du Process Hitting + sortes coopératives

\begin{frame}[t]
  \frametitle{The Process Hitting modeling}
  \framesubtitle{\tcite{\cpmrtcsb}}

% 1 : Sortes
\only<1>{
\tikzstyle{process}=[circle,minimum size=15pt,font=\footnotesize,inner sep=1pt]
\tikzstyle{tick label}=[color=white, font=\footnotesize]
\tikzstyle{tick}=[transparent]
\tikzstyle{hit}=[transparent]
\tikzstyle{selfhit}=[transparent, min distance=30pt,curve to]
\tikzstyle{bounce}=[transparent]
\tikzstyle{hlhit}=[transparent]
\begin{center}\scalebox{\scaleex}{
\begin{tikzpicture}
\exphdef
\end{tikzpicture}
}\end{center}
}

% 2 : Processus
\only<2>{
\tikzstyle{process}=[circle,draw,minimum size=15pt,font=\footnotesize,inner sep=1pt]
\tikzstyle{tick label}=[font=\footnotesize]
\tikzstyle{tick}=[densely dotted]
\tikzstyle{hit}=[transparent]
\tikzstyle{selfhit}=[transparent, min distance=30pt,curve to]
\tikzstyle{bounce}=[transparent]
\tikzstyle{hlhit}=[transparent]
\begin{center}\scalebox{\scaleex}{
\begin{tikzpicture}
\exphdef
\end{tikzpicture}
}\end{center}
}

% 3 : États
\only<3>{
\tikzstyle{hit}=[transparent]
\tikzstyle{selfhit}=[transparent, min distance=30pt,curve to]
\tikzstyle{bounce}=[transparent]
\tikzstyle{hlhit}=[transparent]
\begin{center}\scalebox{\scaleex}{
\begin{tikzpicture}
\exphdef

\TState{3}{a_0,b_1,z_0}
\end{tikzpicture}
}\end{center}
}

% 4 : Actions
\only<4->{
\tikzstyle{tick}=[densely dotted]
\tikzstyle{hit}=[->,>=angle 45]
\tikzstyle{selfhit}=[min distance=30pt,curve to]
\tikzstyle{bounce}=[densely dotted,>=stealth',->]
\tikzstyle{hlhit}=[very thick]
\begin{center}\scalebox{\scaleex}{
\begin{tikzpicture}
\exphdef
\TState{4}{a_0,b_1,z_0}
\TState{5}{a_0,b_1,z_1}
\TState{6}{a_1,b_1,z_1}
\TState{7}{a_1,b_1,z_2}
\end{tikzpicture}
}\end{center}
}

\medskip
\begin{liste}
  \item \tval{Sorts}: components \qex{$a$, $b$, $z$}
\pause[2]
  \item \tval{Processes}: local states / levels of expression \qex{$z_0$, $z_1$, $z_2$}
\pause[3]
  \item \tval{States}: sets of active processes%
  \only<3-4>{\qex{$\PHetat{a_0, b_1, z_0}$}}%
  \only<5>{\qex{$\PHetat{a_0, b_1, z_1}$}}%
  \only<6>{\qex{$\PHetat{a_1, b_1, z_1}$}}%
  \only<7>{\qex{$\PHetat{a_1, b_1, z_2}$}}%
\pause[4]
  \item \tval{Actions}: dynamics \qex{\only<4>{\underline}{$\PHfrappe{b_1}{z_0}{z_1}$}, \only<4-5>{\underline}{$\PHfrappe{a_0}{a_0}{a_1}$}, \only<6>{\underline}{$\PHfrappe{a_1}{z_1}{z_2}$}}
\end{liste}
\end{frame}



\begin{frame}
  \frametitle{Adding cooperations}
  \framesubtitle{\tcite{\cpmrtcsb}}

\begin{center}\scalebox{\scaleex}{
\begin{tikzpicture}
\exphcoop
\end{tikzpicture}
}\end{center}

\medskip
\only<-14>{
\begin{liste}
  \item How to introduce some \tval{cooperation} between sorts? \qex{$\PHfrappe{a_1 \wedge b_0}{z_1}{z_2}$}
\pause[4]
  \item Solution: a \tval{cooperative sort} \qex{$ab$} \only<12->{\quad to express \qex{$a_1 \wedge b_0$}}
\pause[8]
  \item Constraint: each configuration is represented by one process \qex{$\PHetat{a_1,b_0} \pause[11]\Rightarrow ab_{10}$}
\pause[14]
  \item Advantage: regular sort; drawbacks: complexity, temporal shift
\end{liste}}
\end{frame}



% Analyse statique

\subsubsection{Successive reachability}

\begin{frame}
  \frametitle{Static analysis: successive reachability}
  \framesubtitle{\tcite{\cpmrmscs}}

Successive reachability of processes:

\begin{columns}
\begin{column}{0.55\textwidth}

\begin{center}
\scalebox{0.75}{
\begin{tikzpicture}
%\path[use as bounding box] (-1,-3) rectangle (7,2);
\exatt

\TState{2-4}{a_0,b_0,b_2,c_0,d_0}

\TState{5}{a_0,b_0,c_0,d_0}
\TState{6}{a_0,b_0,c_1,d_0}
\TState{7}{a_0,b_0,c_1,d_1}
\TState{8}{a_0,b_1,c_1,d_1}
\TState{9}{a_0,b_1,c_1,d_2}

\node<3>[process,very thick] (d_1) at (d_1.center) {1?};
\node<3>[process,very thick] (b_1) at (b_1.center) {2?};
\node<3>[process,very thick] (d_2) at (d_2.center) {3?};

\node<4-8>[process,very thick] (d_2) at (d_2.center) {1?};
\node<9>[process,very thick] (d_2) at (d_2.center) {};

\only<5>{\THit{a_0}{hlhit}{c_0}{.north}{c_1}}
\path<5>[bounce,bend left,hlhit] \TBounce{c_0}{}{c_1}{.west};
\only<6>{\THit{b_0}{hlhit}{d_0}{.west}{d_1}}
\path<6>[bounce,bend left,hlhit] \TBounce{d_0}{}{d_1}{.south};
\only<7>{\THit{c_1}{bend left=20pt,hlhit}{b_0}{.west}{b_1}}
\path<7>[bounce,bend left,hlhit] \TBounce{b_0}{}{b_1}{.south};
\only<8>{\THit{b_1}{hlhit}{d_1}{.west}{d_2}}
\path<8>[bounce,bend left,hlhit] \TBounce{d_1}{}{d_2}{.south};
\end{tikzpicture}
}
\end{center}

\end{column}
\begin{column}{0.45\textwidth}

\pause
~\\~\\~\\~\\
\begin{itemize}
  \item Initial context
    \\ \rex{\PHetat{a_1, \{b_0, b_1\}, c_0, z_0}} \pause
  \item Objectives
    \\ \rex{$[\ \Rsh d_1 \PHconcat\ \Rsh b_1 \PHconcat\ \Rsh d_2\ ]$} \pause
    \\\smallskip \rex{$[\ \Rsh d_2\ ]$} \pause
\end{itemize}

\end{column}
\end{columns}

\medskip
\begin{center}
\f Concretization of the objective = scenario

\ex{
\only<5>{\underline{$\PHfrappe{a_0}{c_0}{c_1}$}}\only<-4,6->{$\PHfrappe{a_0}{c_0}{c_1}$}~\PHconcat~
\only<6>{\underline{$\PHfrappe{b_0}{d_0}{d_1}$}}\only<-5,7->{$\PHfrappe{b_0}{d_0}{d_1}$}~\PHconcat~
\only<7>{\underline{$\PHfrappe{c_1}{b_0}{b_1}$}}\only<-6,8->{$\PHfrappe{c_1}{b_0}{b_1}$}~\PHconcat~
\only<8>{\underline{$\PHfrappe{b_1}{d_1}{d_2}$}}\only<-7,9->{$\PHfrappe{b_1}{d_1}{d_2}$}}
\end{center}

\end{frame}



\begin{frame}
  \frametitle{Over- and Under-approximations}
  \framesubtitle{\tcite{\cpmrmscs}}

Static analysis by abstractions:
\begin{fleches}
  \item Directly checking an objective sequence $R$ is hard
  \item Rather check the approximations $P$ and $Q$, where \tval{$P \Rightarrow R \Rightarrow Q$}:
\end{fleches}

\begin{center}
\scalebox{0.6}{
\figsa
}
\end{center}

\only<-7>{~}
\only<8->{
Polynomial w.r.t.~the number of sorts and \\exponential w.r.t.~the number of processes in each sort
\begin{fleches}
  \item Efficient for big models with few levels of expression
\end{fleches}
}
\end{frame}



\begin{frame}[c]
  \frametitle{Implementation \& Execution times}

\Pint\tval{: Existing free OCaml library}

\medskip
\f Compiler + tools for Process Hitting models

\f Documentation \& examples: \lien{http://processhitting.wordpress.com/}

\pause
\bigskip
\medskip
\tval{Computation time for various reachability analyses:}

\medskip
\small
\begin{tabular}{r||c|c|c|c||c|c|c|}
\hline
\tval{Model} & Sorts & Procs & Actions & States & Biocham$^1$ & libddd$^2$ & \Pint \\\hline
\tval{\ex{egfr20}} & 35 & 196 & 670 & $2^{64}$ & [3s--$\infty$] & [1s--150s] & \tval{0.007s} \\\hline
\tval{\ex{tcrsig40}} & 54 & 156 & 301 & $2^{73}$ & [1s--$\infty$] & [0.6s--$\infty$] & \tval{0.004s} \\\hline
\tval{\ex{tcrsig94}} & 133 & 448 & 1124 & $2^{194}$ & $\infty$ & $\infty$ & \tval{0.030s} \\\hline
\tval{\ex{egfr104}} & 193 & 748 &  2356 & $2^{320}$ &  $\infty$ & $\infty$ & \tval{0.050s}\\\hline
\end{tabular}

\medskip
\quad$^1$ Inria Paris-Rocquencourt/Contraintes\\
\quad$^2$ LIP6/Move

\cmodels
\end{frame}




\begin{frame}[c]
  \frametitle{The Process Hitting modeling}

\begin{itemize}
  \item \tval{Dynamic} modeling with an \tval{atomistic} point of view
  \begin{fleches}
    \item Independent actions
    \item Cooperation modeled with cooperative sorts
  \end{fleches}

  \smallskip
  \item Efficient \tval{static analysis}
  \begin{fleches}
    \item Reachability of a process can be computed in \tval{polynomial time}\\
          \quad in the number of sorts
  \end{fleches}

  \smallskip
  \item Useful for the study of \tval{large biological models}
  \begin{fleches}
    \item Up to hundreds of sorts
  \end{fleches}

  \smallskip
  \item (Future) extensions
  \begin{fleches}
    \item Actions with priorities
    \item Continuous time with clocks?
  \end{fleches}
\end{itemize}

\end{frame}

