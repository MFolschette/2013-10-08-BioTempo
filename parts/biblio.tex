% Bibliographie

\begin{frame}[c]
  \frametitle{Bibliography}

\footnotesize
\setlength{\parindent}{-1em}
\setlength{\parskip}{0.5em}
~

\vfill

\tcite{PMR10-TCSB} Loïc Paulevé, Morgan Magnin, Olivier Roux. \ex{Refining dynamics of gene regulatory networks in a stochastic $\pi$-calculus framework}. In Corrado Priami, Ralph-Johan Back, Ion Petre, and Erik de Vink, editors: \textit{Transactions on Computational Systems Biology XIII}, volume 6575 of Lecture Notes in Computer Science, 171-191. Springer Berlin/Heidelberg, 2011.

\tcite{PMR12-MSCS} Loïc Paulevé, Morgan Magnin, Olivier Roux. \ex{Static analysis of biological regulatory networks dynamics using abstract interpretation}. \textit{Mathematical Structures in Computer Science}, 2012.

%\tcite{RCB08} Adrien Richard, Jean-Paul Comet, Gilles Bernot. \ex{R. Thomas' logical method}, 2008. Invited at \textit{Tutorials on modelling methods and tools: Modelling a genetic switch and Metabolic Networks}, Spring School on Modelling Complex Biological Systems in the Context of Genomics.

\tcite{RCB06} Adrien Richard, Jean-Paul Comet, Gilles Bernot. \textit{Modern Formal Methods and App.}, chapter \ex{Formal Methods for Modeling Biological Regulatory Networks}, pages 83--122. 2006.


\tcite{CMSB12} Maxime Folschette, Loïc Paulevé, Katsumi Inoue, Morgan Magnin, Olivier Roux. \ex{Concretizing the Process Hitting into Biological Regulatory Networks}. In David Gilbert and Monika Heiner, editors, \textit{Computational Methods in Systems Biology X}, Lecture Notes in Computer Science, pages 166–186. Springer Berlin
Heidelberg, 2012.

%\tcite{Paulevé11} Loïc Paulevé. PhD thesis: \ex{\textit{Modélisation, Simulation et Vérification des Grands Réseaux de Régulation Biologique}}, October 2011, Nantes, France

%\tcite{PMR10-TSE} Loïc Paulevé, Morgan Magnin, and Olivier Roux. \textit{Tuning Temporal Features within the Stochastic $\pi$-Calculus}. IEEE Transactions on Software Engineering, 37(6):858-871, 2011.

%\tcite{PR10-CRAS} Loïc Paulevé and Adrien Richard. \textit{Topological Fixed Points in Boolean Networks}. Comptes Rendus de l'Académie des Sciences - Series I - Mathematics, 348(15-16):825 - 828, 2010.

\vfill
\Large
\begin{flushright}
  \tval{Thank you}\hspace{1cm}~
\end{flushright}
\vfill

~

\end{frame}
